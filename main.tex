\documentclass[a4paper]{article}

%% Language and font encodings
\usepackage[english]{babel}
\usepackage[utf8x]{inputenc}
\usepackage[T1]{fontenc}

%% Sets page size and margins
\usepackage[a4paper,top=3cm,bottom=2cm,left=3cm,right=3cm,marginparwidth=1.75cm]{geometry}

%% Useful packages
\usepackage{mathptmx,amsmath,colonequals}
\usepackage{amssymb,gensymb,mathrsfs}
\usepackage{amscd,amsthm}
\usepackage{mathtools}
\usepackage{graphicx}
\usepackage[colorinlistoftodos]{todonotes}
\usepackage[colorlinks=true, allcolors=blue]{hyperref}
\usepackage[shortlabels]{enumitem}
\usepackage{extarrows}
\usepackage{tikz,tikz-cd}
%\usepackage{hyperref}

\newtheorem{thm}{Theorem}[section]
\newtheorem{lem}[thm]{Lemma}
\newtheorem{cor}[thm]{Corollary}
\newtheorem{prop}[thm]{Proposition}
\newtheorem{conj}[thm]{Conjecture}

%\theoremstyle{definition}
\newtheorem{defn}[thm]{Definition}
\newtheorem{ex}[thm]{Example}
\newtheorem{xca}[thm]{Exercise}

%\theoremstyle{remark}
\newtheorem{rmk}[thm]{Remark}

\newcommand{\arxiv}[1]{\href{http://arxiv.org/abs/#1}{{\tt arXiv:#1}}}

% convenient renaming
\newcommand{\isom}{\cong}
\newcommand{\ins}{\subset}
\newcommand{\dual}{\vee}
\newcommand{\cross}{\times}

% general text operators
\newcommand{\Image}{\operatorname{Im}}
\newcommand{\Coim}{\operatorname{Coim}}
\newcommand{\Ker}{\operatorname{Ker}}
\newcommand{\Coker}{\operatorname{Coker}}
\newcommand{\colim}{\operatorname{colim}}

% arrows
\newcommand{\surj}{\twoheadrightarrow}
\newcommand{\inj}{\xhookrightarrow{}}

% context
\newcommand{\basefield}{K}
\newcommand{\basering}{R}

% algebra
\newcommand{\category}[1]{\textsf{#1}}
\newcommand{\Hom}{\operatorname{Hom}}
\newcommand{\Ext}{\operatorname{Ext}}
\newcommand{\Tor}{\operatorname{Tor}}
% TODO: put a \newcommand for the characteristic

% commutative algebra
%\newcommand{\depth}{\operatorname{depth}}
%\newcommand{\height}{\operatorname{height}}
%\newcommand{\pd}{\operatorname{pd}}
%\newcommand{\injd}{\operatorname{injd}}

% algebraic geometry
\newcommand{\Spec}{\operatorname{Spec}}
\newcommand{\Spa}{\operatorname{Spa}}

% custom ops here
\newcommand{\codim}{\operatorname{codim}}

% custom commands here
\newcommand{\conormal}{\mathscr{I} / \mathscr{I}^{2}}

\title{A detailed proof of the adjunction formula}
\author{Jack J Garzella}

\begin{document}

\maketitle

\section{Regular Varieties}

There are two (very closely related)
notions of ``smoothness'' given in Vakil Chatper 12.
The one we need for this proof is ``regularity''.

\begin{prop}
	Let \((R,\mathfrak{m},k)\) be a noetherian ring. 
	The following are equivalent:
	\begin{enumerate}[(a)]
		\item Let \(n\) be the minimal number of generators of
			\(\mathfrak{m}\).
			Then \(n = \dim R\).
		\item \(\dim \mathfrak{m} / \mathfrak{m}^{2}  = \dim R\)
	\end{enumerate}
\end{prop}


\begin{defn}
	Let \(R\) be a noetherian local ring.
	\(R\) is \textit{regular} if one (equivalently, all)
	of the conditions from the previous proposition holds.
\end{defn}

The part that is needed in user6's proof is (b) in the proposition.

\begin{lem}
	The localization of a regular local ring at (read: away from) a prime 
	ideal is a regular local ring.
\end{lem}


%TODO: define a regular scheme using this as the local definition

%% Smooth morphisms
%
%\begin{prop}
%	Let \(f: X \to Y\) 
%	be a map of schemes which is locally of finite
%	presentation (and thus locally of finite type).
%	The following are equivalent
%	\begin{enumerate}[(i)]
%		\item (Wikipedia Definition)
%			\(f\) is flat and for every geometric point
%			\(s : \overline{k} \to Y\), 
%			the fiber \(X_{s} := X \times_{Y} s\) 
%			is regular.
%		\item (Hartshorne Definition)
%			\(f\) is flat and the sheaf of relative
%			differentials \(\Omega_{X / Y}\) is 
%			locally free of rank equal to the 
%			relative dimension of \(f\).
%		\item (Vakil Definition)
%			For any \(x \in X\), there exists a neighborhood 
%			\(\Spec B\) of \(x\) and a neighborhood 
%			\(\Spec A\) of \(f(x)\) such that 
%			\(B = A[t_{1}, \ldots, t_{n}] / (P_{1}, \ldots, P_{m})\) 
%			and the Jacobian matrix has full rank, i.e.
%			the ideal generated by the \(m\)-by-\(m\) minors
%			of \((\partial P_{i} / \partial t_{j})\) is \(B\).
%			
%		\item \(f\) is formally smooth
%	\end{enumerate}
%\end{prop}
%
%\begin{def}
%	If any of the previous conditions is satisfied, 
%	the map \(f\) is said to be a \textit{smooth} morphism
%	of schemes.
%\end{def}
%
%\begin{def}
%	A \(k\)-scheme \(X\) is  \textit{smooth over \(k\)} if 
%	the structure map \(X \to k\) is smooth.
%\end{def}

\section{Sheaf of Kahler differentials}

\begin{itemize}
	\item define the module of kahler differentials
		(Vakil has 3 definitions, we'll need to figure out 
		how to do this correctly).
    \item affine conormal exact sequence
		(algebraic computation using the definition)
	\item define the sheaf of kahler differentials
		(globalizing the previous)
	\item sheafy conormal exact sequence
		(globalizing the previous)
\end{itemize}


We will need ideal sheaves for this.

\section{Noetherian Schemes}

\begin{defn}
	Let \(X\) be a scheme. 
	We say \(X\) is \textit{locally noetherian}
	if there exists an open affine cover
	\(\{U_{i} = \Spec A_{i}\}\) such that
	all \(A_{i}\) are noetherian rings.
\end{defn}




\section{Regularity implies conormal sequence is left-exact}

\begin{thm}
	[Hartshorne Exercise II.5.7a]
	Let \(X\) be a locally noetherian scheme, and let
	\(\mathcal{F}\) be a coherent sheaf. 
	If the stalk \(\mathcal{F}_{x}\) is a free
	\(\mathcal{O}_{X,x} \)-module for some point
	\(x \in X\), then there exists
	a neighborhood \(U\) containing \(x\) such that
	\(\mathcal{F}|_{U}\) is free.
\end{thm}

\begin{proof}
	WLOG, we can assume that \(X\) is affine, so 
	\(X = \Spec A\), and furthermore we can assume
	\(A\) is noetherian.
	Indeed, take an open affine noetherian cover 
	as guarenteed by local noetherianity. 
	Then \(x\) is contained in some neighborhood
	\(U = \Spec A\) in the cover. 
	% this is a (re)definiton of \Spec A, a slight abuse
	%   of notation
	If we show the theorem for \(\Spec A\),
	then we have shown it for \(X\).

	Now, as \(\mathcal{F}\) is a coherent sheaf on 
	\(A\), \(\mathcal{F} \isom \tilde{M}\) for some
	finitely generated module \(M\) on \(A\).
	\(x\), being a point of \(\Spec A\), 
	is/corresponds to a prime
	ideal \(\mathfrak{p}\) in \(A\).
	Now, our assumption about
	freeness of the stalk says that
	\(M_{\mathfrak{p}} \isom A_{\mathfrak{p}}^{\oplus n}\) 
	for some \(n\).
	Indeed, \(M_{\mathfrak{p}}\) is the stalk of 
	\(\tilde{M}\) and \(A_{\mathfrak{p}}\) is the local
	ring of \(\Spec A\) at \(\mathfrak{p}\).
	Let \(m_{1} , \ldots , m_{n}\) the a 
	basis / free generating set for \(M_{\mathfrak{p}}\).
	In other words, \(m_{i}\) is the image 
	(along
	some isomorphism \(A_{\mathfrak{p}}^{\oplus n}\)) of
	\((0, \ldots, 1, \ldots, 0) \in A_{\mathfrak{p}}^{\oplus n}\)
	where the \(1\) is in the \(i\)-th place.
	Let \(x_{1}, \ldots, x_{k}\) be a finite generating set
	for \(M\).
	
	Now, we have the following equations in \(M_{\mathfrak{p}}\):
	\[
		\frac{x_{i}}{1} = \sum_{j=1}^{n} \frac{a_{ij}}{b_{ij}} m_{j} 
	.\] 
	Aside: if \(\frac{x_{i}}{1} = 0\), then all the \(a_{ij}\) are zero
	and the \(b_{ij}\) are \(1\) and the rest of the proof is
	not affected.

	Now, we make the following claim: 
	The \(m_{i}\) generate 
	
	% pause here, confusion about what my source is saying.


\end{proof}


\begin{thm}
	[Hartshorne II.8.15]
	Let \(X\) be an irreducible separated scheme of finite type 
	over an algebraically closed field \(k\).
	Then \(\Omega_{X / k}\) is a locally free
	sheaf of rank \(\dim X\) if and only if 
	\(X\) is regular.
\end{thm}


\begin{thm}
	[Hartshorne II.8.17]
	Let \(X\) be a regular variety over \(k\).
	Let \(W\) be an irreducible closed subscheme 
	with corresponding sheaf of ideals \(\mathscr{I}\).
	Then \(W\) is regular if and only if the following
	two conditions hold:
	\begin{enumerate}[(1)]
		\item \(\Omega_{Y / k}\) is locally free
		\item The conormal exact sequence
			\[
			\begin{tikzcd}
			0 \arrow{r}{} & \conormal \arrow{r}{} & 
			\Omega_{X / k} \otimes \mathcal{O}_{Y}  \arrow{r}{} & 
			\Omega_{Y / k} \arrow{r}{} & 0
			\end{tikzcd}
			\]
			is exact.
	\end{enumerate}
\end{thm}

\begin{cor}
	In the situation of the conclusion of the previous theorem, 
	\(\mathscr{I}\) is locally generated by 
	\(\codim(W)\) elements.
\end{cor}

\begin{cor}
	In the situation of the conclusion of the previous theorem,
	\(\conormal\) is locally free of rank \(\codim(W)\).
\end{cor}


%There are a few of this that I know of:
%
%Hartshorne's proof, which needs
%* nakayama's lemma (in mathlib)
%* integral schemes
%* a proper closed integral subscheme has codimension at least one
%https://math.stackexchange.com/questions/2372649/nakayamas-lemma-in-theorem-8-17-of-chapter-ii-in-hartshorne
%https://math.stackexchange.com/questions/3327995/some-fine-details-in-the-proof-of-hartshorne-ii-8-17
%* also requires Hartshorne II.5.7, Hartshorne II.8.8, Hartshorne II.8.7
%
%Vakil's proof, which needs
%* associated primes
%* geometric interpretation of associated primes
%
%There are also three other proofs from stack exchange,
%https://math.stackexchange.com/questions/846346/hartshorne-theorem-8-17
%
%Ignacio Barros's proof:
%* passes to stalks
%* needs Hartshorne II.8.7, which should be doable with just 
%   facts about differentials
%
%Tomo's proof:
%* Uses the concept of \textit{formally smooth} morphisms
%
%user6's proof:
%* passes to stalks, uses properties of free modules
%* uses Nakayama's lemma
%* requires Hartshorne II.5.7, which is proved here:
%https://dornsife.usc.edu/assets/sites/618/docs/Hartshorne\_Exercises.pdf
%* this also requires the following theorem:
%\begin{thm}
%	Let \(f: M \to M\) be a surjective endomorphism of \(R\)-modules.
%	Then \(R\) is an isomoprhism.
%\end{thm}
%The proof uses Nakayama's lemma, and this is precisely where we need it.


% conormal exact sequence

% Sketch of Hartshorne's proof:

% stuff => smooth
%  use nakayama's lemma, then compute the dimension of the z cotan space
% smooth => stuff
%  pick sub sheaf/variety for which it works on stalks, by first direction
%  this is regular and irreducible.
%  Then use the universal property integral closure (both schemes are integral)
%  (probably using integral <==> reduced and irred)

\section{Alternate description of the conormal sheaf}

\begin{itemize}
	\item restriction of sheaves
	\item adjunction of pushforward of sheaves and pullback of sheaves
	\item ideal sheaves
	\item ``passing to stalks''
\end{itemize}

The proof of this is detailed in a nice amount of detail in
https://math.stackexchange.com/questions/1672117/conormal-bundle-of-cartier-divisors

Note that in the stack exchange article they prove more or less this 
theorem for an arbitrary subvariety, and the part about divisors
is only because of the alternate description of divisors as ideal sheaves.

\section{The determinant bundle}


\section{The adjunction formula}

% the end goal

\begin{thm}[Adjunction Formula]
	Let \(X\) be a smooth variety and \(D\) a
	(smooth?) divisor. 
	Then 
	\[
		(K_{X} + D)|_{D} = K_{D}
	\] 
	
\end{thm}



\end{document}
