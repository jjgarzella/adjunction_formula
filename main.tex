\documentclass[a4paper]{article}

%% Language and font encodings
\usepackage[english]{babel}
\usepackage[utf8x]{inputenc}
\usepackage[T1]{fontenc}

%% Sets page size and margins
\usepackage[a4paper,top=3cm,bottom=2cm,left=3cm,right=3cm,marginparwidth=1.75cm]{geometry}

%% Useful packages
\usepackage{mathptmx,amsmath,colonequals}
\usepackage{amssymb,gensymb,mathrsfs}
\usepackage{amscd,amsthm}
\usepackage{mathtools}
\usepackage{graphicx}
\usepackage[colorinlistoftodos]{todonotes}
\usepackage[colorlinks=true, allcolors=blue]{hyperref}
\usepackage[shortlabels]{enumitem}
\usepackage{extarrows}
\usepackage{tikz,tikz-cd}
%\usepackage{hyperref}

\newtheorem{thm}{Theorem}[section]
\newtheorem{lem}[thm]{Lemma}
\newtheorem{cor}[thm]{Corollary}
\newtheorem{prop}[thm]{Proposition}
\newtheorem{conj}[thm]{Conjecture}

%\theoremstyle{definition}
\newtheorem{defn}[thm]{Definition}
\newtheorem{ex}[thm]{Example}
\newtheorem{xca}[thm]{Exercise}

%\theoremstyle{remark}
\newtheorem{rmk}[thm]{Remark}

\newcommand{\arxiv}[1]{\href{http://arxiv.org/abs/#1}{{\tt arXiv:#1}}}

% convenient renaming
\newcommand{\isom}{\cong}
\newcommand{\ins}{\subset}
\newcommand{\dual}{\vee}
\newcommand{\cross}{\times}

% general text operators
\newcommand{\Image}{\operatorname{Im}}
\newcommand{\Coim}{\operatorname{Coim}}
\newcommand{\Ker}{\operatorname{Ker}}
\newcommand{\Coker}{\operatorname{Coker}}
\newcommand{\colim}{\operatorname{colim}}

% arrows
\newcommand{\surj}{\twoheadrightarrow}
\newcommand{\inj}{\xhookrightarrow{}}

% context
\newcommand{\basefield}{K}
\newcommand{\basering}{R}

% algebra
\newcommand{\category}[1]{\textsf{#1}}
\newcommand{\Hom}{\operatorname{Hom}}
\newcommand{\Ext}{\operatorname{Ext}}
\newcommand{\Tor}{\operatorname{Tor}}
% TODO: put a \newcommand for the characteristic

% commutative algebra
%\newcommand{\depth}{\operatorname{depth}}
%\newcommand{\height}{\operatorname{height}}
%\newcommand{\pd}{\operatorname{pd}}
%\newcommand{\injd}{\operatorname{injd}}

% algebraic geometry
\newcommand{\Spec}{\operatorname{Spec}}
\newcommand{\Spa}{\operatorname{Spa}}

% custom ops here

% custom commands here

\title{A detailed proof of the adjunction formula}
\author{Jack J Garzella}

\begin{document}

\maketitle

\section{Regular Varieties}

There are two (very closely related)
notions of ``smoothness'' given in Vakil Chatper 12.
The one we need for this proof is ``regularity''.

\begin{prop}
	Let \((R,\mathfrak{m},k)\) be a noetherian ring. 
	The following are equivalent:
	\begin{enumerate}[(a)]
		\item Let \(n\) be the minimal number of generators of
			\(\mathfrak{m}\).
			Then \(n = \dim R\).
		\item \(\dim \mathfrak{m} / \mathfrak{m}^{2}  = \dim R\)
	\end{enumerate}
\end{prop}


\begin{defn}
	Let \(R\) be a noetherian local ring.
	\(R\) is \textit{regular} if one (equivalently, all)
	of the conditions from the previous proposition holds.
\end{defn}

%TODO: define a regular scheme using this as the local definition

%% Smooth morphisms
%
%\begin{prop}
%	Let \(f: X \to Y\) 
%	be a map of schemes which is locally of finite
%	presentation (and thus locally of finite type).
%	The following are equivalent
%	\begin{enumerate}[(i)]
%		\item (Wikipedia Definition)
%			\(f\) is flat and for every geometric point
%			\(s : \overline{k} \to Y\), 
%			the fiber \(X_{s} := X \times_{Y} s\) 
%			is regular.
%		\item (Hartshorne Definition)
%			\(f\) is flat and the sheaf of relative
%			differentials \(\Omega_{X / Y}\) is 
%			locally free of rank equal to the 
%			relative dimension of \(f\).
%		\item (Vakil Definition)
%			For any \(x \in X\), there exists a neighborhood 
%			\(\Spec B\) of \(x\) and a neighborhood 
%			\(\Spec A\) of \(f(x)\) such that 
%			\(B = A[t_{1}, \ldots, t_{n}] / (P_{1}, \ldots, P_{m})\) 
%			and the Jacobian matrix has full rank, i.e.
%			the ideal generated by the \(m\)-by-\(m\) minors
%			of \((\partial P_{i} / \partial t_{j})\) is \(B\).
%			
%		\item \(f\) is formally smooth
%	\end{enumerate}
%\end{prop}
%
%\begin{def}
%	If any of the previous conditions is satisfied, 
%	the map \(f\) is said to be a \textit{smooth} morphism
%	of schemes.
%\end{def}
%
%\begin{def}
%	A \(k\)-scheme \(X\) is  \textit{smooth over \(k\)} if 
%	the structure map \(X \to k\) is smooth.
%\end{def}

\section{Sheaf of Kahler differentials}

\begin{itemize}
	\item define the module of kahler differentials
	\item define the sheaf of kahler differentials
	\item prove that the conormal exact sequence is right-exact
\end{itemize}

\section{The conormal (left)-exact sequence}

We will need ideal sheaves for this.

\section{Milestone 3: Regularity implies conormal sequence is left-exact}

There are a few of this that I know of:

Hartshorne's proof, which needs
* nakayama's lemma (in mathlib)
* integral schemes
* a proper closed integral subscheme has codimension at least one
https://math.stackexchange.com/questions/2372649/nakayamas-lemma-in-theorem-8-17-of-chapter-ii-in-hartshorne
https://math.stackexchange.com/questions/3327995/some-fine-details-in-the-proof-of-hartshorne-ii-8-17

Vakil's proof, which needs
* associated primes
* geometric interpretation of associated primes

There are also three other proofs from stack exchange,
https://math.stackexchange.com/questions/846346/hartshorne-theorem-8-17

Ignacio Barros's proof:
* passes to stalks
* needs Hartshorne II.8.7, which should be doable with just 
   facts about differentials

Tomo's proof:
* Uses the concept of \textit{formally smooth} morphisms

user6's proof:
* passes to stalks, uses properties of free modules
* uses Nakayama's lemma

% conormal exact sequence

% Sketch of Hartshorne's proof:

% stuff => smooth
%  use nakayama's lemma, then compute the dimension of the z cotan space
% smooth => stuff
%  pick sub sheaf/variety for which it works on stalks, by first direction
%  this is regular and irreducible.
%  Then use the universal property integral closure (both schemes are integral)
%  (probably using integral <==> reduced and irred)

\section{Milestone 4: Alternate description of the conormal sheaf}

\begin{itemize}
	\item restriction of sheaves
	\item adjunction of pushforward of sheaves and pullback of sheaves
	\item ideal sheaves
	\item ``passing to stalks''
\end{itemize}

The proof of this is detailed in a nice amount of detail in
https://math.stackexchange.com/questions/1672117/conormal-bundle-of-cartier-divisors

Note that in the stack exchange article they prove more or less this 
theorem for an arbitrary subvariety, and the part about divisors
is only because of the alternate description of divisors as ideal sheaves.

\section{Milestone 5: The adjunction formula}



% the end goal

\begin{thm}[Adjunction Formula]
	Let \(X\) be a smooth variety and \(D\) a
	(smooth?) divisor. 
	Then 
	\[
		(K_{X} + D)|_{D} = K_{D}
	\] 
	
\end{thm}



\end{document}
